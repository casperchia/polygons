\documentclass[12pt,titlepage]{article}

\usepackage{lastpage}
\usepackage{fancyhdr}
\usepackage[usenames,dvipsnames]{color}
\usepackage{tabularx}
\usepackage{geometry}
\usepackage{parskip}

\headheight 15pt
\cfoot{\textbf{\thepage}\ of \textbf{\pageref{LastPage}}}

\newenvironment{storyPoint}
   {\color{blue} \bfseries}
   {}

\newenvironment{storyId}
   {\color{Bittersweet} \itshape}
   {}

\newenvironment{optionalStory}
   {\color[rgb]{0.015625,0.2890625,0.046875}}
   {}

\begin{document}

\title{
   Project Plan \\
   COMP4920 -- 2014s2
}

\author{
   Andrew Cai -- z3290228 \\
   \and
   Casper Chia -- z3413140 \\
   \and 
   Matthew Gino Ciancio -- z3374281 \\
   \and
   Luna Pradhananga -- z3358423 \\
   \and
   Corey Tattam -- z3221450
}

\date{\today}

\maketitle

\newpage
\tableofcontents
\newpage

\section{Project Vision}
Produce a web application that allows users to interactively plan their CSE
degrees based on degree/course constraints and to also allow users to contribute
to publicly viewable course reviews/evaluations.

\subsection{Degree Planner}
Interaction with the user begins by getting them to choose from a list of
CSE degrees/programs offered by UNSW. Upon making a selection, a list is
produced which contains all the mandatory courses that are required to complete
the degree, where each course is allocated in a degree time slot (e.g. Semester
2, Year 2). Where there are gaps between the mandatory courses, a user can fill
them in from a list that would be appropriate to the time slot and prerequisite
constraints.

\subsection{Course Reviewer}
Each course offered by UNSW will be listed on the web application and have it's
own review page which consists of aggregated ratings (e.g. fun factor, challenge
factor, etc.) and more detailed reviews all submitted by users. I.e. A user can
select a course and submit a review that will be aggregated onto the course
review page and contribute to the aid of students deciding on what courses
they would like to study.

\subsection{Misc}
While these two features are useful in themselves, when combined, they do an
excellent job at helping a future student choose a CSE degree that they will
enjoy and hopefully get the most out of. I.e. They will be more informed about
all levels of a degree, like never before.

\section{Epics}

Note: All epics and user stories are written from the perspective of an
end-user, more specifically a student/prospective student user.

% Define epics here in this hacky way, so that they can be spewed out as many 
% times as needed throughout the document.
\newcommand{\epicOne}{
   Be able to visit a website in my web browser with an appealing interface.
}
\newcommand{\epicTwo}{
   Be able to choose the CSE degree I am interested in studying at UNSW.
}
\newcommand{\epicThree}{
   See what courses I have to take in order to complete a degree.
}
\newcommand{\epicFour}{
   Observe what electives I can choose throughout my degree and add them into my
   degree plan.
}
\newcommand{\epicFive}{
   Save my degree so that I can come back and edit it later on.
}
\newcommand{\epicSix}{
   Export my degree plan as a PDF document.
}
\newcommand{\epicSeven}{
   Browse through the list of courses offered by UNSW and to be able to sort and
   filter them in various ways.
}
\newcommand{\epicEight}{
   View the review page of a course which contains various forms of ratings and
   detailed reviews of the course, submitted by previous students.
}
\newcommand{\epicNine}{
   Be able to submit a review of a course I have previously studied and have it
   added to the pool of reviews for a course (i.e. my review should appear on
   the course review page).
}
\newcommand{\epicTen}{
   I should be able to edit any part of any review I have ever submitted, at any
   time in the future and the course review page should be updated accordingly.
}
\newcommand{\epicEleven}{
   Users should be able to create an account, but have to supply an email
   address with a ``.*unsw.edu.au'' domain and follow the activation link sent
   to that email, in order to confirm they are a UNSW student.
}
\newcommand{\epicTwelve}{
   There should be a user profile page that allows users to edit their profile
   details, lists all their degree plans and all their course reviews.
}
\newcommand{\epicThirteen}{
   I should be able to plan other UNSW degrees beyond the school of Computer
   Science and Engineering.
}

\begin{enumerate}
   \item \epicOne{}
   \item \epicTwo{}
   \item \epicThree{} 
   \item \epicFour{}
   \item \epicFive{}
   \item \epicSix{}
   \item \epicSeven{} 
   \item \epicEight{}
   \item \epicNine{}
   \item \epicTen{}
   \item \epicEleven{} 
   \item \epicTwelve{}
   \item \epicThirteen{}
\end{enumerate}

\section{User Stories}

\begin{storyId}
Note: The text in this style after each user story denotes the ID of the user
story.
\end{storyId}

\begin{storyPoint}
Note: The text that appears with this style at the beginning of each user story
is the assigned story points.
\end{storyPoint}

\begin{optionalStory}
Note: The text that appears with this style, indicates that a user story is not
mandatory for the successful release of the system/project.
\end{optionalStory}

\subsection{\epicOne{}}

% Define user stories here in this hacky way, so that they can be spewed out
% as many times as needed throughout the document.
\newcommand{\storyOnePointOne}{
   \begin{storyPoint}1\end{storyPoint}
   Visit a URL and be taken to the website.
   \begin{storyId}1.1\end{storyId}
}
\newcommand{\storyOnePointTwo}{
   \begin{storyPoint}5\end{storyPoint}
   Website should be aesthetically pleasing, have a unified interface and be
   easy to use.
   \begin{storyId}1.2\end{storyId}
}
\newcommand{\storyOnePointThree}{
   \begin{storyPoint}1\end{storyPoint}
   Ability to access degree planner, course list and course review pages all
   without having to be logged into an account.
   \begin{storyId}1.3\end{storyId}
}
\newcommand{\storyOnePointFour}{
   \begin{storyPoint}1\end{storyPoint} 
   Home page should summarise the website and provide links to the various
   features/pages.
   \begin{storyId}1.4\end{storyId}
}

\begin{enumerate}
   \item \storyOnePointOne{}
   \item \storyOnePointTwo{}
   \item \storyOnePointThree{}
   \item \storyOnePointFour{}
\end{enumerate}

\subsection{\epicTwo{}}

\newcommand{\storyTwoPointOne}{
   \begin{storyPoint}1\end{storyPoint}
   Degree planner page should be accessible from all pages.
   \begin{storyId}2.1\end{storyId}
}
\newcommand{\storyTwoPointTwo}{
   \begin{storyPoint}2\end{storyPoint}
   First degree planner page should present me with a list of CSE degrees to
   choose from.
   \begin{storyId}2.2\end{storyId}
}
\newcommand{\storyTwoPointThree}{
   \begin{storyPoint}3\end{storyPoint}
   I should be able to view all the degrees at once, or filtered degrees by
   letter, or filtered degrees by faculty.
   \begin{storyId}2.3\end{storyId}
}

\begin{enumerate}
   \item \storyTwoPointOne{}
   \item \storyTwoPointTwo{}
   \item \storyTwoPointThree{}
\end{enumerate}

\subsection{\epicThree{}}

\newcommand{\storyThreePointOne}{
   \begin{storyPoint}3\end{storyPoint}
   Upon selecting a degree from the list, a page should be displayed which
   contains a link to the handbook entry for the degree and a list of compulsory
   courses for the degree.
   \begin{storyId}3.1\end{storyId}
}
\newcommand{\storyThreePointTwo}{
   \begin{storyPoint}4\end{storyPoint}
   Courses should be ordered by time slot (e.g. Semester 2, Year 2) and there
   should be gaps between compulsory courses (for where electives are to go).
   \begin{storyId}3.2\end{storyId}
}
\newcommand{\storyThreePointThree}{
   \begin{storyPoint}2\end{storyPoint}
   Each course should have its code, name and hyperlink to its review page.
   \begin{storyId}3.3\end{storyId}
}
\newcommand{\storyThreePointFour}{
   \begin{storyPoint}10\end{storyPoint}
   There should be stand out text that displays whether the current degree plan
   is eligible for graduation status or not, including the number of units of
   credit that need to be added.
   \begin{storyId}3.4\end{storyId}
}
\newcommand{\storyThreePointFive}{
   \begin{storyPoint}2\end{storyPoint}
   Mandatory courses should not be able to be removed from the plan.
   \begin{storyId}3.5\end{storyId}
}

\begin{enumerate}
   \item \storyThreePointOne{}
   \item \storyThreePointTwo{}
   \item \storyThreePointThree{}
   \item \storyThreePointFour{}
   \item \storyThreePointFive{}
\end{enumerate}

\subsection{\epicFour{}}

\newcommand{\storyFourPointOne}{
   \begin{storyPoint}3\end{storyPoint}
   In each elective slot there should be a button to insert an elective, which
   when clicked displays a list of courses that can be taken in that time frame.
   \begin{storyId}4.1\end{storyId}
}
\newcommand{\storyFourPointTwo}{
   \begin{storyPoint}7\end{storyPoint}
   The elective course list should be populated by whether or not prerequisites
   have been met at that stage, if the course is offered in that time period and
   if it isn't already in the plan.
   \begin{storyId}4.2\end{storyId}
}
\newcommand{\storyFourPointThree}{
   \begin{storyPoint}4\end{storyPoint}
   The courses should be able to be filtered alphabetically and by subject area
   (e.g. COMP for Computer Science courses).
   \begin{storyId}4.3\end{storyId}
}
\newcommand{\storyFourPointFour}{
   \begin{storyPoint}2\end{storyPoint}
   When a course is selected it should be added to the plan in the appropriate
   time slot.
   \begin{storyId}4.4\end{storyId}
}
\newcommand{\storyFourPointFive}{
   \begin{storyPoint}2\end{storyPoint}
   Non-compulsory courses in the plan should have a little ``X'' button appear
   next to them when hovered and when pressed the course should be removed from
   the plan.
   \begin{storyId}4.5\end{storyId}
}

\begin{enumerate}
   \item \storyFourPointOne{}
   \item \storyFourPointTwo{}
   \item \storyFourPointThree{}
   \item \storyFourPointFour{}
   \item \storyFourPointFive{}
\end{enumerate}

\subsection{\epicFive{}}

\newcommand{\storyFivePointOne}{
   \begin{storyPoint}1\end{storyPoint}
   There should be a save/permalink button on the degree plan page.
   \begin{storyId}5.1\end{storyId}
}
\newcommand{\storyFivePointTwo}{
   \begin{storyPoint}2\end{storyPoint}
   When the button is pressed, a widget/window should appear saying that I can
   come back to this plan later on via the URL in this widget/window.
   \begin{storyId}5.2\end{storyId}
}
\newcommand{\storyFivePointThree}{
   \begin{storyPoint}3\end{storyPoint}
   If the user is logged in, the save/permalink button should tell them the plan
   has been saved to their account/profile.
   \begin{storyId}5.3\end{storyId}
}
\newcommand{\storyFivePointFour}{
   \begin{storyPoint}5\end{storyPoint}
   Degree plans should be saved indefinitely, even for non-logged in users.
   \begin{storyId}5.4\end{storyId}
}
\newcommand{\storyFivePointFive}{
   \begin{storyPoint}3\end{storyPoint}
   For logged in users, their latest degree plan will not be saved to their
   account unless they press the save/permalink button. Between the time when
   they create the plan and when they save it, the plan is treated like an
   anonymous user's plan (i.e. it is saved indefinitely, but only accessible
   via the permalink).
   \begin{storyId}5.5\end{storyId}
}
\newcommand{\storyFivePointSix}{
   \begin{storyPoint}5\end{storyPoint}
   If a logged in user created plan is not saved within a week, it should be
   deleted from the system, but not before a warning email is sent to the user,
   containing a permalink to the plan so that they can save it before it is
   deleted.
   \begin{storyId}5.6\end{storyId}
}
\newcommand{\storyFivePointSeven}{
   \begin{storyPoint}4\end{storyPoint}
   Each degree plan page should have a delete button, which when pressed should
   present a confirmation widget/window and then permanently delete the plan
   from the system if the user confirms to do so.
   \begin{storyId}5.7\end{storyId}
}

\begin{enumerate}
   \item \storyFivePointOne{}
   \item \storyFivePointTwo{}
   \item \storyFivePointThree{}
   \item \storyFivePointFour{}
   \item \storyFivePointFive{}
   \item \storyFivePointSix{}
   \item \storyFivePointSeven{}
\end{enumerate}

\subsection{\epicSix{}}

\newcommand{\storySixPointOne}{
   \begin{storyPoint}1\end{storyPoint}
   There should be a button on every degree plan page that says something along
   the lines of ``Export as PDF''.
   \begin{storyId}6.1\end{storyId}
}
\newcommand{\storySixPointTwo}{
   \begin{storyPoint}5\end{storyPoint}
   When pressed, a new browser tab should be opened and a PDF of the current
   degree plan should be present in the tab (allowing the user to save it to
   their computer if they wish).
   \begin{storyId}6.2\end{storyId}
}

\begin{enumerate}
   \item \storySixPointOne{}
   \item \storySixPointTwo{}
\end{enumerate}

\subsection{\epicSeven{}}

\newcommand{\storySevenPointOne}{
   \begin{storyPoint}1\end{storyPoint}
   There should be a link to the UNSW course reviews from every page.
   \begin{storyId}7.1\end{storyId}
}
\newcommand{\storySevenPointTwo}{
   \begin{storyPoint}4\end{storyPoint}
   The course review page should contain a description of what the course
   reviews are all about, a link to all UNSW courses, set of alphabetical links
   and a set of specialisation links (e.g. COMP for Computer Science).
   \begin{storyId}7.2\end{storyId}
}
\newcommand{\storySevenPointThree}{
   \begin{storyPoint}2\end{storyPoint}
   When the ``All'' link is pressed, all UNSW courses should be presented.
   \begin{storyId}7.3\end{storyId}
}
\newcommand{\storySevenPointFour}{
   \begin{storyPoint}2\end{storyPoint}
   When one of the alphabetic links is pressed, all courses according to the
   letter pressed should be displayed.
   \begin{storyId}7.4\end{storyId}
}
\newcommand{\storySevenPointFive}{
   \begin{storyPoint}2\end{storyPoint}
   When one of the specialisation links is pressed, all courses according to the
   specialisation should be displayed.
   \begin{storyId}7.5\end{storyId}
}
\newcommand{\storySevenPointSix}{
   \begin{storyPoint}2\end{storyPoint}
   Each course listing (on the various listing pages) should display the course
   code, name and be a hyperlink to the individual course review page.
   \begin{storyId}7.6\end{storyId}
}

\begin{enumerate}
   \item \storySevenPointOne{}
   \item \storySevenPointTwo{}
   \item \storySevenPointThree{}
   \item \storySevenPointFour{}
   \item \storySevenPointFive{}
   \item \storySevenPointSix{}
\end{enumerate}

\subsection{\epicEight{}}

\newcommand{\storyEightPointOne}{
   \begin{storyPoint}1\end{storyPoint}
   Each course review page should contain the course code, name and link to its
   handbook entry.
   \begin{storyId}8.1\end{storyId}
}
\newcommand{\storyEightPointTwo}{
   \begin{storyPoint}3\end{storyPoint}
   There should be displays of aggregated review results: interest, fun,
   challenging, recommendable, encouraged you to think critically and education
   on the topic.
   \begin{storyId}8.2\end{storyId}
}
\newcommand{\storyEightPointThree}{
   \begin{storyPoint}5\end{storyPoint}
   There should be a list of Lecturers whom have taken the course and when the
   user clicks on a lecturer, displays of aggregated results should be presented
   for that lecturer: engaging, humble, experience, clarity and dedication.
   \begin{storyId}8.3\end{storyId}
}
\newcommand{\storyEightPointFour}{
   \begin{storyPoint}5\end{storyPoint}
   Similar to the previous requirement, but for tutors.
   \begin{storyId}8.4\end{storyId}
}
\newcommand{\storyEightPointFive}{
   \begin{storyPoint}7\end{storyPoint}
   There should also be a display of the aggregated overall rating (out of five)
   of the course, e.g. a graph of the course ratings.
   \begin{storyId}8.5\end{storyId}
}
\newcommand{\storyEightPointSix}{
   \begin{storyPoint}3\end{storyPoint}
   Users should be able to view a list of descriptive reviews of the course,
   filtered by overall course rating.
   \begin{storyId}8.6\end{storyId}
}
\newcommand{\storyEightPointSeven}{
   \begin{storyPoint}1\end{storyPoint}
   All reviews should be displayed anonymously, i.e. no identifiable information
   shall be displayed next to each descriptive review.
   \begin{storyId}8.7\end{storyId}
}

\begin{enumerate}
   \item \storyEightPointOne{}
   \item \storyEightPointTwo{}
   \item \storyEightPointThree{}
   \item \storyEightPointFour{}
   \item \storyEightPointFive{}
   \item \storyEightPointSix{}
   \item \storyEightPointSeven{}
\end{enumerate}

\subsection{\epicNine{}}

\newcommand{\storyNinePointOne}{
   \begin{storyPoint}1\end{storyPoint}
   On a course review page, there should a button that says something along the
   lines of ``Submit Your Review''.
   \begin{storyId}9.1\end{storyId}
}
\newcommand{\storyNinePointTwo}{
   \begin{storyPoint}1\end{storyPoint}
   If an anonymous user presses the button they should be redirected to the
   login page, otherwise if the user is logged in, they should be taken to the
   course review submission page.
   \begin{storyId}9.2\end{storyId}
}
\newcommand{\storyNinePointThree}{
   \begin{storyPoint}6\end{storyPoint}
   The course review submission page should be very similar to the corresponding
   course review page, except the displays should be replaced with editable
   fields/widgets.
   \begin{storyId}9.3\end{storyId}
}
\newcommand{\storyNinePointFour}{
   \begin{storyPoint}3\end{storyPoint}
   There should be a submit button which when pressed should confirm with the
   user if they wish to submit the review and inform them that they can always
   edit it later by coming to the course review page or clicking the link on
   their profile page (under the list of submitted reviews).
   \begin{storyId}9.4\end{storyId}
}

\begin{enumerate}
   \item \storyNinePointOne{}
   \item \storyNinePointTwo{}
   \item \storyNinePointThree{}
   \item \storyNinePointFour{}
\end{enumerate}

\subsection{\epicTen{}}

\newcommand{\storyTenPointOne}{
   \begin{storyPoint}2\end{storyPoint}
   If a logged in user has submitted a review for a course, the course review
   page should not contain a ``Submit Your Review'' button, but should instead
   contain a ``Edit Your Review'' page which should take them to the course
   review submission page, which contains all their previously saved work.
   \begin{storyId}10.1\end{storyId}
}

\begin{enumerate}
   \item \storyTenPointOne{}
\end{enumerate}

\subsection{\epicEleven{}}

\newcommand{\storyElevenPointOne}{
   \begin{storyPoint}1\end{storyPoint}
   There should be a login button/link present on all pages when a user is not
   logged in.
   \begin{storyId}11.1\end{storyId}
}
\newcommand{\storyElevenPointTwo}{
   \begin{storyPoint}1\end{storyPoint}
   The login page should contain a link that says something along the lines of
   ``Don't have an account yet?'' and takes them to the account creation page
   when clicked.
   \begin{storyId}11.2\end{storyId}
}
\newcommand{\storyElevenPointThree}{
   \begin{storyPoint}2\end{storyPoint}
   Account creation page should ask for various details: UNSW email address and
   password. I.e. if they supply an email address without a ``*unsw.edu.au''
   domain, account creation will be denied.
   \begin{storyId}11.3\end{storyId}
}
\newcommand{\storyElevenPointFour}{
   \begin{storyPoint}1\end{storyPoint}
   The page should inform users that all reviews that they submit with their
   account will be anonymous to others.
   \begin{storyId}11.4\end{storyId}
}
\newcommand{\storyElevenPointFive}{
   \begin{storyPoint}2\end{storyPoint}
   When a user creates their account, an email should be sent to the supplied
   UNSW email address containing an activation link which the user must visit in
   order for them to login into and use their account.
   \begin{storyId}11.5\end{storyId}
}
\newcommand{\storyElevenPointSix}{
   \begin{storyPoint}4\end{storyPoint}
   Inactivated accounts should be deleted after a week, not before another
   activation email is sent to the user after the first one, also telling them
   their account will be deleted if they do not activate it.
   \begin{storyId}11.6\end{storyId}
}

\begin{enumerate}
   \item \storyElevenPointOne{}
   \item \storyElevenPointTwo{}
   \item \storyElevenPointThree{}
   \item \storyElevenPointFour{}
   \item \storyElevenPointFive{}
   \item \storyElevenPointSix{}
\end{enumerate}

\subsection{\epicTwelve{}}

\newcommand{\storyTwelvePointOne}{
   \begin{storyPoint}1\end{storyPoint}
   A button/link should exist on every page for logged in users, which when
   pressed, takes them to their profile page.
   \begin{storyId}12.1\end{storyId}
}
\newcommand{\storyTwelvePointTwo}{
   \begin{storyPoint}2\end{storyPoint}
   The page should contain a form allowing them to change their password and
   their email. Changing their email will result in them have to confirm the
   UNSW email address again and if it is not confirmed within half an hour, the
   old email address will not be updated.
   \begin{storyId}12.2\end{storyId}
}
\newcommand{\storyTwelvePointThree}{
   \begin{storyPoint}2\end{storyPoint}
   The profile page should also contain a list of degree plans the user has
   saved and links to view/edit them. 
   \begin{storyId}12.3\end{storyId}
}
\newcommand{\storyTwelvePointFour}{
   \begin{storyPoint}2\end{storyPoint}
   The page should also contain a list of the reviews the user has submitted and
   links to pages where they can be edited.
   \begin{storyId}12.4\end{storyId}
}

\begin{enumerate}
   \item \storyTwelvePointOne{}
   \item \storyTwelvePointTwo{}
   \item \storyTwelvePointThree{}
   \item \storyTwelvePointFour{}
\end{enumerate}

\subsection{\epicThirteen{}}

\newcommand{\storyThirteenPointOne}{
   \begin{storyPoint}20\end{storyPoint}
   \begin{optionalStory}
   The degree planner should be “extended” to allow users to create plans for
   all UNSW degrees.
   \end{optionalStory}
   \begin{storyId}13.1\end{storyId}
}

\begin{enumerate}
   \item \storyThirteenPointOne{}
\end{enumerate}

\section{Product Backlog}

User stories in the backlog are ordered by their priority to complete, where
high priority stories are at the top and vice versa.

\begin{enumerate}
   \item \storyOnePointOne{}
   \item \storyOnePointTwo{}
   \item \storyOnePointFour{}
   \item \storyTwoPointTwo{}
   \item \storyTwoPointThree{} 
   \item \storyThreePointOne{}
   \item \storyFivePointFour{}
   \item \storyThreePointThree{} 
   \item \storyThreePointTwo{}
   \item \storyThreePointFive{}
   \item \storyFourPointOne{}
   \item \storyFourPointTwo{}
   \item \storyFourPointThree{} 
   \item \storyFourPointFour{}
   \item \storyFourPointFive{}
   \item \storyFivePointOne{}
   \item \storyFivePointTwo{}
   \item \storyThreePointFour{}
   \item \storySevenPointTwo{}
   \item \storySevenPointThree{} 
   \item \storySevenPointFour{}
   \item \storySevenPointFive{}
   \item \storySevenPointSix{}
   \item \storyEightPointOne{}
   \item \storyEightPointTwo{}
   \item \storyEightPointThree{} 
   \item \storyEightPointFour{}
   \item \storyEightPointFive{}
   \item \storyEightPointSix{}
   \item \storyEightPointSeven{} 
   \item \storyNinePointTwo{}
   \item \storyOnePointThree{} 
   \item \storyTwoPointOne{}
   \item \storySevenPointOne{} 
   \item \storyElevenPointOne{} 
   \item \storyElevenPointTwo{}
   \item \storyElevenPointThree{} 
   \item \storyElevenPointFour{}
   \item \storyElevenPointFive{}
   \item \storyFivePointThree{} 
   \item \storyFivePointFive{}
   \item \storyNinePointOne{}
   \item \storyNinePointThree{} 
   \item \storyNinePointFour{}
   \item \storyTenPointOne{}
   \item \storyTwelvePointOne{}
   \item \storyTwelvePointTwo{}
   \item \storyTwelvePointThree{} 
   \item \storyTwelvePointFour{}
   \item \storySixPointOne{}
   \item \storySixPointTwo{}
   \item \storyFivePointSeven{} 
   \item \storyFivePointSix{}
   \item \storyElevenPointSix{}
   \item \storyThirteenPointOne{}
\end{enumerate}

\section{Release Plan}

\newgeometry{left=1cm,right=1cm}
\begin{table}[H]
\centering
\def\tabularxcolumn#1{m{#1}}
\begin{tabularx}{0.9\paperwidth}{|c|X|c|c|c|c|}
\hline
\textbf{\#} & \textbf{Description} & \textbf{Release Type} & \textbf{Sprints} &
\textbf{Start Date} & \textbf{Release Date} \\
\hline
1 & Have basic website up and running. Able to choose degrees (within CSE) and
    display compulsory courses. & Major & 1 & 22/9/2014 & 28/9/2014 \\
\hline
2 & Able to view, add and remove elective courses. Also able to check if current
    plan is eligible for graduation status. & Major & 2 & 29/9/2014 & 5/10/2014
    \\
\hline
3 & Course review system is complete. Users can create accounts to post reviews
    and save degree plans. & Significant & 3-4 & 6/10/2014 & 19/10/2014 \\
\hline
4 & Degree planner now caters for most if not all UNSW degrees. & Major & 5 &
20/10/2014 & 26/10/2014 \\
\hline
\end{tabularx}
\end{table}
\restoregeometry

\subsection{Conditions of Satisfaction}
Using test-driven development means that there will always be unit tests for
code and therefore we can use them as an automated means of ensuring that things
are working as they should and that we are progressing as planned.

In addition to the automated unit testing, end-users of the system should always
be able to operate and engage with the web application to check whether their
requirements/stories have been fulfilled. \\*
More specifically, we will have a physical production server and domain name
which will be used to host a live version of the system for end-users to provide
feedback on.

\section{Sprints}
Total story point tally = 171 \\*
Number of sprints = 5 \\*
Sprint story point capacity = ceiling(171 / 5) = 35

Thus each sprint should ideally take around 35 story points, in the planned
timetable. The ceiling was taken in order to spread the fraction of work over
all sprints, as opposed to loading up the last sprint.

\subsection{Sprint \#1 (Story point tally = 38)}
\textbf{Schedule: 22nd September - 28th September}
\begin{enumerate}
   \item \storyOnePointOne{}
   \item \storyOnePointTwo{}
   \item \storyOnePointFour{}
   \item \storyTwoPointTwo{}
   \item \storyTwoPointThree{}
   \item \storyThreePointOne{}
   \item \storyFivePointFour{}
   \item \storyThreePointThree{}
   \item \storyThreePointTwo{}
   \item \storyThreePointFive{}
   \item \storyFourPointOne{}
   \item \storyFourPointTwo{}
\end{enumerate}

\subsection{Sprint \#2 (Story point tally = 37)}
\textbf{Schedule: 29th September - 5th October}
\begin{enumerate}
   \item \storyFourPointThree{}
   \item \storyFourPointFour{}
   \item \storyFourPointFive{}
   \item \storyFivePointOne{}
   \item \storyFivePointTwo{}
   \item \storyThreePointFour{}
   \item \storySevenPointTwo{}
   \item \storySevenPointThree{}
   \item \storySevenPointFour{}
   \item \storySevenPointFive{}
   \item \storySevenPointSix{}
   \item \storyEightPointOne{}
   \item \storyEightPointTwo{}
\end{enumerate}

\subsection{Sprint \#3 (Story point tally = 35)}
\textbf{Schedule: 6th October - 12th October}
\begin{enumerate}
   \item \storyEightPointThree{}
   \item \storyEightPointFour{}
   \item \storyEightPointFive{}
   \item \storyEightPointSix{}
   \item \storyEightPointSeven{}
   \item \storyNinePointTwo{}
   \item \storyOnePointThree{}
   \item \storyTwoPointOne{}
   \item \storySevenPointOne{}
   \item \storyElevenPointOne{}
   \item \storyElevenPointTwo{}
   \item \storyElevenPointThree{}
   \item \storyElevenPointFour{}
   \item \storyElevenPointFive{}
   \item \storyFivePointThree{}
\end{enumerate}

\subsection{Sprint \#4 (Story point tally = 32)}
\textbf{Schedule: 13th October - 19th October}
\begin{enumerate}
   \item \storyFivePointFive{}
   \item \storyNinePointOne{}
   \item \storyNinePointThree{}
   \item \storyNinePointFour{}
   \item \storyTenPointOne{}
   \item \storyTwelvePointOne{}
   \item \storyTwelvePointTwo{}
   \item \storyTwelvePointThree{}
   \item \storyTwelvePointFour{}
   \item \storySixPointOne{}
   \item \storySixPointTwo{}
   \item \storyFivePointSeven{}
\end{enumerate}

\subsection{Sprint \#5 (Story point tally = 29)}
\textbf{Schedule: 20th October - 26th October}
\begin{enumerate}
   \item \storyFivePointSix{}
   \item \storyElevenPointSix{}
   \item \storyThirteenPointOne{}
\end{enumerate}

\section{Allocation of Roles}

\textbf{Scrum Master}: Matt \\*
\textbf{Product Owner}: Andrew \\*
\textbf{Developers}: Andrew, Casper, Corey, Luna, Matt \\*
\textbf{Developmental Testers}: Andrew, Casper, Corey, Luna, Matt \\*

\section{Meeting Schedule}

\subsection{Daily Stand-up}
Stand-up meetings have been organised to start from anywhere between 9-10pm every
week night. \\*
These meetings will be extremely brief and simply consist of everyone mentioning
how they are going with the developmental work. Any issues will be raised and
the scrum master will take notes and use them to ensure the problems get
resolved ASAP.

\subsection{Sprint Planning/Review/Retrospective}
The goal here is to cut down on having three separate meetings, because they
coincide so closely to each other (temporally) and it makes sense to merge them
into a single meeting. \\*
These meetings will be scheduled for 11am Monday mornings and will involve
everyone discussing how they went with the previous sprint. \\*
Stories will be rearranged between sprints as necessary, to compensate for being
ahead/behind schedule. Feedback will also be given in regard to how well the
previous sprint went, processes that need improving for future sprints and any
other problems that should arise.

\section{Extreme Programming}

\subsection{Test-Driven Development}
All developers will write their own test cases, preferably before they write
code and the test cases will be added to the project-wide suite of unit test
cases.

Ideally developers should write test cases for other developers' code, but we
need to be realistic and realise this is not practical given our circumstances
(small time frame, external commitments, geographic differences, etc.) and would
severely hinder team velocity. Having said that, if during end-user manual
acceptance testing, bugs are spotted, then we will zero-in on the affected code
and developers who did not write the code will write test cases for it, followed
by the patching of the bug.

All unit tests will be run daily to ensure that no regressions occur.

\end{document}
