\documentclass[12pt]{article}

\usepackage[margin=1cm]{geometry}
\usepackage{color}

\pagenumbering{gobble}

\newenvironment{command}
   { 
      \begin{quote}\itshape
      \color{blue}
   }
   { \end{quote} }

\newenvironment{data}
   { 
      \begin{quote}\itshape
      \color{red}
   }
   { \end{quote} }

\begin{document}

\section{Introduction}

\begin{command} Note that text in this style denotes commands that need to be
executed within a console/terminal. \end{command}

\begin{data} Note that text in this style denotes data that should be entered or
pasted into the instructed file. \end{data}

Replace all mentions of "\$PARENT\_DIR" with the path where your clone of the
repository will be kept. I.e. not the repo path itself, but it's parent
directory path.

Replace all mentions of "\$PROJECT\_ROOT" with the path of the repository, e.g.
it should look something like "\$PARENT\_DIR/polygons/", assuming you haven't
renamed the directory that the repository was cloned into.

\section{Steps}

\begin{enumerate}
   \item \begin{command} cd \$PROJECT\_ROOT/src \end{command}
   \item \begin{command} cp ../misc/unsw\_db.tar.bz2 . \end{command}
   \item \begin{command} tar -xvf unsw\_db.tar.bz2 \end{command}
   \item Skip the next 4 steps if your polygons database is already up to date.
   \item \begin{command} dropdb polygons \end{command}
   \item \begin{command} createdb -U postgres polygons \end{command}
   \item \begin{command} ./manage.py migrate \end{command}
   \item \begin{command} ./core\_data.py \end{command}
   \item \begin{command} createdb unsw \end{command}
   \item \begin{command} psql unsw -f dump.sql \end{command}
   \item \begin{command} pg\_dump -a -t acad\_object\_groups -t courses -t
                         program\_degrees -t orgunit\_groups -t orgunits -t
                         program\_group\_members -t program\_rules -t programs
                         -t rules -t stream\_group\_members -t stream\_rules -t
                         streams -t subject\_group\_members -t subject\_prereqs
                         -t subjects unsw \textgreater unsw\_data\_old.sql \end{command}
   \item \begin{command} cp ../utils/unsw\_db\_converter.py . \end{command}
   \item \begin{command} ./unsw\_db\_converter.py unsw\_data\_old.sql
                         \textgreater unsw\_data\_new.sql\end{command}
   \item \begin{command} psql polygons -f unsw\_data\_new.sql \end{command}
\end{enumerate}

\end{document}
