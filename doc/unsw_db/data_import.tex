\documentclass[12pt]{article}

\usepackage[margin=1cm]{geometry}
\usepackage{color}

\pagenumbering{gobble}

\newenvironment{command}
   { 
      \begin{quote}\itshape
      \color{blue}
   }
   { \end{quote} }

\newenvironment{data}
   { 
      \begin{quote}\itshape
      \color{red}
   }
   { \end{quote} }

\begin{document}

\section{Introduction}

\begin{command} Note that text in this style denotes commands that need to be
executed within a console/terminal. \end{command}

\begin{data} Note that text in this style denotes data that should be entered or
pasted into the instructed file. \end{data}

Replace all mentions of "\$PARENT\_DIR" with the path where your clone of the
repository will be kept. I.e. not the repo path itself, but it's parent
directory path.

Replace all mentions of "\$PROJECT\_ROOT" with the path of the repository, e.g.
it should look something like "\$PARENT\_DIR/polygons/", assuming you haven't
renamed the directory that the repository was cloned into.

\section{Steps}

\begin{enumerate}
   \item Untar the UNSW database backup archive.
   \item Create a clean "polygons" database if need be.
   \item Ensure "polygons" database is up to date by run running the "manage.py
         migrate" command then running the core\_data.py script.
   \item Copy the utils/unsw\_db\_converter.py script into the src/ directory.
   \item Run the converter script, passing in an argument which is the path to
         the UNSW database dump file and redirect the output to an .sql file.
   \item \begin{command} psql polygons -f pathToConvertedDump.sql \end{command}
   \item Your polygons database should now be filled with the UNSW data.
\end{enumerate}

\end{document}
