\documentclass[12pt]{article}

\usepackage[margin=1cm]{geometry}
\usepackage{color}

\pagenumbering{gobble}

\newenvironment{command}
   { 
      \begin{quote}\itshape
      \color{blue}
   }
   { \end{quote} }

\newenvironment{data}
   { 
      \begin{quote}\itshape
      \color{red}
   }
   { \end{quote} }

\begin{document}

\section{Introduction}
This document will provide step-by-step instructions on how to configure the
COMP4920 Polygons web application stack. \\*
The instructions assume you are running the Ubuntu 14.04 (or some other
Debian-dervied) operating system. \\*

\begin{command} Note that text in this style denotes commands that need to be
executed within a console/terminal. \end{command}

\begin{data} Note that text in this style denotes data that should be entered or
pasted into the instructed file. \end{data}

Replace all mentions of "\$PARENT\_DIR" with the path where your clone of the
repository will be kept. I.e. not the repo path itself, but it's parent
directory path.

Replace all mentions of "\$PROJECT\_ROOT" with the path of the repository, e.g.
it should look something like "\$PARENT\_DIR/polygons/", assuming you haven't
renamed the directory that the repository was cloned into.

\section{Steps}

\begin{enumerate}
   \item \begin{command} sudo apt-get install git \end{command}
   \item \begin{command} cd \$PARENT\_DIR \end{command}
   \item \begin{command} git clone "git clone
         https://bitbucket.org/mciancio/polygons.git" \end{command}
   \item \begin{command} sudo apt-get install python-pip \end{command}
   \item \begin{command} sudo pip install django-lockout \end{command}
   \item \begin{command} sudo apt-get install apache2 \end{command}
   \item Copy the polygons.conf file into /etc/apache2/sites-available/
   \item \begin{command} sudo apt-get install libapache2-mog-wsgi \end{command}
   \item Add the following line to /etc/apache2/envvars:
         \begin{data}
         export POLYGONS\_ROOT=\$PROJECT\_ROOT/src/
         \end{data}
   \item Add the following line to /etc/apache2/ports.conf:
         \begin{data}
         Listen 9000
         \end{data}
   \item \begin{command} sudo service apache2 restart \end{command}
   \item \begin{command} sudo pip install Django==1.7 \end{command}
   \item \begin{command} sudo apt-get install postgresql \end{command}
   \item \begin{command} sudo apt-get install python-psycopg2 \end{command}
   \item \begin{command} sudo su - postgres \end{command}
   \item \begin{command} createdb polygons \end{command}
   \item \begin{command} exit \end{command}
   \item Edit the file /etc/postgresql/9.3/main/pg\_hba.conf and ensure there is
         a line in it like so (note there is no \# at the begnning):
         \begin{data}
         local all postgres trust
         \end{data}
   \item \begin{command} sudo service postgresql restart \end{command}
   \item \begin{command} cd \$PROJECT\_ROOT/src \end{command}
   \item \begin{command} ./manage.py migrate \end{command}
\end{enumerate}

If you now visit http://localhost:9000/ in your web browser, you should
hopefully see a hello world message, which indicates a successful setup.

\end{document}
